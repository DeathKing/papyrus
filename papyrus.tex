%!TEX program = xelatex
\documentclass[a4paper,UTF8]{ctexart}
%-----------------------导言区结开始-------------------------------
\usepackage{etex}
\usepackage{fontspec}
\usepackage{amsmath}              % AMS 符号、环境 
\usepackage{amssymb}              % AMS 符号
\usepackage{float}               
\usepackage{graphicx}             % 用于引入图片
\usepackage{epstopdf}             % 对 eps 图片的支持
\usepackage{mathrsfs}             % \mathscr 支持
\usepackage{stmaryrd}             % 部分符号支持
\usepackage{color}                % 颜色宏包
\usepackage{ulem}                 % 用于下划线的编排
\usepackage{lmodern}              % 使用 Computer Modern 字体
\usepackage[margin=2cm]{geometry} % 用于修改文稿整体格式
\usepackage{titlesec}             % 用于为标题指定样式,仅在单个文章整理成册时使用
\usepackage{tikz}                 % tikz 绘图宏包
\usepackage[colorlinks]{hyperref} % 用户编排超链接
\usepackage[noblocks]{authblk}    % 用于编排文章标题下方的作者信息
\usepackage[export]{adjustbox}    % 为图片提供 center 选项
\usepackage{shortvrb}             % || as \verb|| after \MakeShortVerb
\usepackage{bytefield}            % 用于编排内存布局
\usepackage[toc,page]{appendix}   % 用于编排附录

%==================================================================
% TeXpad 请使用词条命令
\usepackage[texpad]{papyrus}
%------------------------------------------------------------------
% 其它软件用户请使用下述命令
% \usepackage{papyrus}
%==================================================================

%-----------------------导言区结束-----------------------------------

%-----------------------文 档开始-----------------------------------
\begin{document}

%==================================================================
% 封面
%==================================================================
\newcommand{\pubno}{000}    % TR编号,按照十进制三位数编写
\newcommand{\pubtitle}{papyrus 模板介绍及使用说明\par Ver 0.3 } % TR 标题
\newcommand{\pubyear}{2016} % 发表年份
\newcommand{\pubmonth}{10}  % 发表月份
\newcommand{\pubday}{25}    % 发表日

\begin{titlepage}
	\centering
	\par\vspace{5cm}\par
 	\includegraphics[width=0.5\textwidth]{hit_logo}
	\par\vspace{1cm}
	{\scshape\LARGE Harbin Institute of Technology\par}
	\vspace{1cm}
	{\scshape\Large Lilac TechReprot  \textnumero .\pubno\par}
	\vspace{1.5cm}
	{\huge\bfseries \pubtitle \par}
	\vspace{2cm}
	
	%=======================================================
	% 原著请用下面的方式
	% {\Large\itshape 作者1 \: 作者2 \par}
	% 译著请用下面的方式
	{\Large\itshape 作者1 \: 作者2 \quad  \bfseries 著\par}
	{\Large\itshape 作者1 \: 作者2 \quad \bfseries 译\par}
	% 如果没有指导老师,请注释掉一下三行
	\vfill
	{\bfseries 指导老师}\par
	{\itshape 指导老师1 \: 指导老师2}
	%=======================================================

	\vfill
	{\large \pubyear 年 \pubmonth 月 \pubday 日\par}
\end{titlepage}

% 文章内部标题默认与封面保持一致 如果有特殊需要可以自行重载
\title{\pubtitle}

% 作者及单位,推荐用户查看 authblk 宏包手册
% 作者之间的分隔符
\renewcommand{\Authsep}{, }
% 当有多个作者时,下面的命令可以去掉多个作者之间的“and”
\renewcommand{\Authand}{}
\renewcommand{\Authands}{}

\author{7HxzZ}
\affil{哈尔滨工业大学 Lilac战队}

\author{作者2}
\affil{银河工业大学}

% \authorcr 可以在新一行排版作者
\author{\authorcr 作者3}
\affil{宇宙工业大学}

\maketitle

%================================摘要部分================================
% 摘要部分 如果不需要,则可注释此部分
\begin{center}
\parbox{0.9\textwidth}{
\textbf{摘~~~要}\quad 摘要是对文章的内容不加注释和评论的简短陈述,要求扼要地说明研究工作的目的、研究方法和最终结论等。本模板目前是用一种非常ad-hoc方法实现排版摘要,用户只需要在相关地方输入相关文字即可。\\
\textbf{关键词}\quad 关键词, 要用, 英文逗号, 隔开\\}
\end{center}
%================================摘要结束================================

%--------------------------------正文开始--------------------------------
\section{Papyrus 模板简述}

本文档是哈尔滨工业大学 7HxzZ 战队 WriteUp \LaTeX 写作模板 papyrus 的简单简介。虽然使用 Markdown 写作比较方便,但对于一些高级格式,Markdown 无法正确排版。另外,如果想要将文档整理成适合印刷的出版物,Markdown 可能就有点力不从心了。故此,整理了一份 \LaTeX 格式的 WriteUp 写作模板。本模板不但可用于排版单个 WriteUp ,还可用来排版相关技术报告或科技论文等,具有极高的实用价值。

papyrus是“羊皮卷”的意思,本模板的特色有:

\begin{enumerate}
	\item 可以单独按照文章编译(\code/ctexart/类),也可以将所有 WriteUp 文章整理成册,按照书籍编译(\code/ctexbook/类)。两者无需改动文章源码。
	\item 提供了 \code/\code/ 命令,可用于编排单行代码(不完美)。
	\item 封装了 \code/minted/ 环境,可以简单地排版出含有代码高亮的代码。
	\item 提供了 \code/quizdesc/ 环境,可用于编排问答模式 CTF 比赛的题目描述。
\end{enumerate}

本模板托管在 \url{https://github.com/DeathKing/papyrus} ,用户可以关注该代码仓库,获得本模板的最新更新。当然,本模板还有许多问题,需要大家多提提意见,最好能发一些 Pull Request。

\section{学习 \LaTeX 排版的相关资源}

跟 Markdown 相比,\LaTeX 可能稍显复杂。但是,如果稍微花费一点时间学习 \LaTeX ,你会惊奇地发现,\LaTeX 是如此地强大!碍于版面的限制,这里简单地罗列一些 \LaTeX 学习资源,读者可以根据自身情况选择合适的资料进行学习。

\begin{enumerate}
	\item 一份不太简短的 \LaTeXe 介绍(或 93 分钟学会 \LaTeXe):\url{http://www.mohu.org/info/lshort-cn.pdf}。这份约莫100页的中文文档,囊括了 \LaTeX 最常用的功能,之中包含了大量的范例代码,简洁易懂。
	\item \LaTeX Tutorials A PRIMER:\url{http://www.tug.org/twg/mactex/tutorials/ltxprimer-1.0.pdf}。这份广受赞誉的 \LaTeX 教程是由印度 \TeX 用户组撰写的,全书用英文写就。
	\item 邹思宇的 \LaTeX 排版手记 :\url{https://github.com/Zousiyu/Study-LaTeX}。这是网友自己整理的 \LaTeX 学习笔记,对初学者来说很有用处。
	\item 一份其实很短的 \LaTeX 入门文档:\url{http://liam0205.me/2014/09/08/latex-introduction/}。这是由 \CTeX 开发者之一撰写的文档,有很高的人气。
	\item More Math Into \LaTeX :\url{http://www.latexstudio.net/wp-content/uploads/2016/09/More_Math_Into_LaTeX-Springer2016.pdf}。这是一本关于 \LaTeX 数学公式排版的详尽参考书,如果你想进一步成为 \LaTeX 排版专家或者使用 \LaTeX 进行严肃的学术写作,那么请仔细阅读本书!
\end{enumerate}


\section{关于papyrus宏包的使用说明}

\subsection{引入papyrus宏包}

本模板的核心是 papyrus 宏包,该宏包使用 minted 宏包编排代码,而该宏包与 TeXpad 软件有部分冲突,本样式文件已修复。请用户在使用本模板时,注意第25行左右的代码:

\begin{minted}[firstnumber=25]{latex}
%==================================================================
% TeXpad 请使用词条命令
\usepackage[texpad]{papyrus}
%------------------------------------------------------------------
% 其它软件用户请使用下述命令
% \usepackage{papyrus}
%==================================================================
\end{minted}

请根据自己使用的编辑器选择合适的命令引入 papyrus 样式包!

\subsection{\texttt{quizdesc}环境的使用}

第一节通常用来简要描述题目的要求,可以用\verb|\begin{quizdesc}...\end{quizdesc}|环境来描述:\\


\begin{quizdesc}[label=Crypto100 Simple]
Become admin!

http://52.69.244.164:51913
simple-01018f60e497b8180d6c92237e2b3a67.rb
\end{quizdesc}

对应的代码是:

\begin{minted}{latex}
\begin{quizdesc}[label=Crypto100 Simple]
Become admin!

http://52.69.244.164:51913
simple-01018f60e497b8180d6c92237e2b3a67.rb
\end{quizdesc}
\end{minted}


\subsection{代码编排}

撰写 WriteUp 时,经常需要插入代码。行内代码请使用\code/\code/\slash \code/inline code/\slash 来编排,例如,代码\code/while (true) { printf("Hello, World!\n"); } /的编排代码如下:

\begin{minted}{latex}
\code/while (true) { printf("Hello, World!\n"); } /
\end{minted}

当然,也可以使用\code/\verb|the-inline-code|/来编排行内代码,这个功能由 Verbtaim 包提供。

由于编排行内代码存在对特殊符号支持不是很好,所以不推荐用单行代码来编排 Flag 。多行代码请使用\code/minted/环境来编排:

\begin{minted}{ruby}
#!/usr/bin/env ruby

class SampleCode
  def initialize
  end
end
\end{minted}

对应的代码是:

\begin{verbatim}
\begin{minted}{ruby}
#!/usr/bin/env ruby

class SampleCode
  def initialize
  end
end
\end{minted}
\end{verbatim}

minted环境对于Tab缩进支持不是很好,推荐使用空格进行缩进。对于大部分代码(特别是Python代码),由于其前面自带的Tab缩进,推荐为\code/minted/环境添加\code/[autogobble]/选项来去除不必要的缩进,或者使用Sublime Text Editor等具有列编辑功能的编辑器快速删除掉缩进。

关于\code/minted/环境的使用方法,请参考:\url{ftp://ftp.dante.de/tex-archive/macros/latex/contrib/minted/minted.pdf}。

\subsection{关于代码高亮 - Pygments}

\code/minted/ 环境是靠一个外部工具 Pygments 实现代码高亮的,因此需要先安装并设置 Pygments 。这里提供几个有用的连接:

\begin{enumerate}
	\item 如何安装(特别是 Pygments ):\url{http://tex.stackexchange.com/questions/108661/how-to-use-minted-under-miktex-and-windows-7}
	\item 支持的语言:\url{http://pygments.org/languages/}
	\item 支持的词法器(这个网页中的Short names就应该是你的语言名称):\url{http://pygments.org/docs/lexers/}
\end{enumerate}

\subsection{图片的编排}

插图是科技论文和 WriteUp 写作的硬需求,很多同学反映用 \LaTeX 插图非常麻烦。其实不然,利用 \code/\includegraphics{}/ 命令,就可以很轻松地插入一张图片。通常来说,我们把这个图片放入一个\code/figure/ 环境下,并使用一个 \code/\caption{}/ 命令为图标标注名称。

\begin{figure}[hbt!]
  \includegraphics[width=0.80\textwidth,center]{hisp2}
  \caption{Hisp语言}\label{hisp}
\end{figure}

图 \ref{hisp} 是Hisp语言的Logo。它对应的代码如下:

\begin{minted}{latex}
\begin{figure}[hbt!]
  \includegraphics[width=0.80\textwidth,center]{hisp2}
  \caption{Hisp语言}\label{hisp}
\end{figure}
\end{minted}

这段代码的解释如下:

\begin{enumerate}
	\item 首先,我们启用了一个 \code/figure/ 环境,图形(figure)环境有一个可选参数项允许用户来指示图形有可能被放置的位置。关于参数的具体意义,请参考\url{http://www.ctex.org/documents/latex/graphics/node64.html}。
	\item 然后,\code/\includegraphics{}/ 命令用于插入一组图片,不用书写图片的扩展名,并且要将图片与 \code/.tex/ 文件存放在同一文件夹。(绝对路径、设定了 \code/\includegraphics{}/ 命令搜索的文件夹的情况下除外。)\code/\includegraphics{}/ 命令还可以接收一组参数,本例中 \code/width=0.80\textwidth/ 就告知 \LaTeX ,将图片按宽度缩放为文本行宽的0.8倍大小。而 \code/center/ 是由 \code{adjustbox} 宏包提供的选项参数,用于将图片居中。
	\item 特别地,可以用\code/\label{}/命令来标记图表或公示,利用\code/\ref{}/命令来引用它们对应的编号。

\end{enumerate}

\subsection{bytefield包简单使用说明}

考虑到渗透和逆向题目的WriteUp中经常需要编排内存布局,本模板的Ver0.2版本引入了\code/bytefield/包,可用于描绘内存布局。图 \ref{fig:hacked} 就是一个使用 \code/bytefield/包绘制的内存布局图,其代码如代码片段 \ref{lst:bf} 所示。


\begin{figure}[!htb]
\begin{center}
\begin{bytefield}{24}
	\wordbox{2}{\code/system()/}\\
	\wordbox{2}{\Verb|"/bin/sh"|}\\
	\wordbox{2}{\code/"pop rdi; ret"/}\\
	\wordbox{2}{EBP}\\
	\wordbox{2}{Canary}\\
	\begin{rightwordgroup}{0x200}
		\wordbox{10}{Buffer}
	\end{rightwordgroup}
\end{bytefield}
\end{center}
\caption{攻击后的栈帧布局}\label{fig:hacked}
\end{figure}

\begin{listing}[!htb]
\begin{minted}{latex}
\begin{figure}[!htb]
\begin{center}
\begin{bytefield}{24}
    \wordbox{2}{\code/system()/}\\
    \wordbox{2}{\Verb|"/bin/sh"|}\\
    \wordbox{2}{\code/"pop rdi; ret"/}\\
    \wordbox{2}{EBP}\\
    \wordbox{2}{Canary}\\
    \begin{rightwordgroup}{0x200}
        \wordbox{10}{Buffer}
    \end{rightwordgroup}
\end{bytefield}
\end{center}
\caption{攻击后的栈帧布局}\label{fig:hacked}
\end{figure}
\end{minted}
\caption{\code/bytefiled/包使用代码示意}\label{lst:bf}
\end{listing}

\subsection{\texttt{ulem}宏包使用方法}

\texttt{ulem} 宏包主要用于编排下划线、删除线,代码及对应的效果如表\ref{tbl:ulem}所示:

\begin{table}[!htb]
\caption{\code/ulem/包编排效果}\label{tbl:ulem}
\begin{center}
\begin{tabular}{l@{\quad}l}\hline\noalign{\vskip2pt}
   \verb|\uline{important}|  & underlined text like \uline{important}\\[1pt]
   \verb|\uuline{urgent}|    & double-underlined text like  \uuline{urgent}\\[1pt]
   \verb|\uwave{boat}|       & wavy underline like {\let\ULleaders\cleaders\uwave{boat}}\\[1pt]
   \verb|\sout{wrong}|       & line struck through word like \sout{wrong}\\[1pt]
   \verb|\xout{removed}|     & marked over like \xout{removed} \\[1pt]
   \verb|\dashuline{dashing}|& dashed underline like \dashuline{dashing}\\[1pt]
   \verb|\dotuline{dotty}|   & dotted underline like \dotuline{dotty}\\[3pt]\hline
\end{tabular}
\end{center}	
\end{table}

\section{其它元素的编排}

\subsection{参考文献的编排}

papyrus 可以快速方便地编排参考文献。例如,本模板中的参考文献编排代码如代码片段\ref{lst:ref}所示,如果想要引用文献,可以使用\code/\cite{the-label-name}/\cite{wiki_aes} 来实现。

\begin{listing}[!htb]
\begin{minted}{latex}
\begin{thebibliography}{99}

\bibitem{wiki_aes} Advanced Encryption Standard, \url{https://en.wikipedia.org/wiki/Advanced_Encryption_Standard}
\bibitem{wiki_bcmo} Block cipher mode of operation, \url{https://en.wikipedia.org/wiki/Block_cipher_mode_of_operation}
\bibitem{sch07} Schneier B. Applied cryptography: protocols, algorithms, and source code in C[M]. john wiley \& sons, 2007.
\end{thebibliography}
\end{minted}
\caption{参考文献的编排}\label{lst:ref}
\end{listing}

\subsection{附录的编排}

对于一些行数较多的代码或代码文件,推荐将内容编排到附录中,并在正文中引用该章节。附录的编排请参考\code/papyrus.tex/源码。

%------------------------------ 正文结束 --------------------------------

%------------------------------ 参考文献 --------------------------------
\small
\begin{thebibliography}{99}

\bibitem{wiki_aes} Advanced Encryption Standard, \url{https://en.wikipedia.org/wiki/Advanced_Encryption_Standard}
\bibitem{wiki_bcmo} Block cipher mode of operation, \url{https://en.wikipedia.org/wiki/Block_cipher_mode_of_operation}
\bibitem{sch07} Schneier B. Applied cryptography: protocols, algorithms, and source code in C[M]. john wiley \& sons, 2007.
\end{thebibliography}
%----------------------------- 文献结束 ---------------------------------

%----------------------------- 附  录 ---------------------------------

%=======================================================================
% 如果想要紧凑的布局(节省纸张),请使用下面的代码,附录与参考文献保持一个合理的距离
\par \vspace{1cm}
% 如果想要使附录从新一页开始,请注释掉上一行,并启用下面这行代码
%\newpage
%=======================================================================

\appendix

% 对于书籍的编排 如果要把附录放入目录 请启用下面的代码
% \addcontentsline{toc}{chapter}{附录}

{\par \noindent \huge \bfseries \appendixname \par}

\section{附录1}

这里可以放置附录。

\section{附录2}

这里可以放置附录。
%----------------------------- 附录结束 ---------------------------------

\end{document}
